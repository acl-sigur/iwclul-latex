\documentclass[11pt]{article}
\usepackage{times}
\sloppy
\hyphenpenalty 10000

% for Letter size

%\setlength\topmargin{0.2cm} \setlength\oddsidemargin{-0cm}
%\setlength\textheight{22cm} \setlength\textwidth{15.8cm}
%\setlength\columnsep{0.25in}  \newlength\titlebox \setlength\titlebox{2.00in}
%\setlength\headheight{5pt}   \setlength\headsep{0pt}
%%\setlength\footheight{0pt}
%\setlength\footskip{1.0cm}
%\setlength\leftmargin{0.0in}
%%\thispagestyle{empty}
%\pagestyle{empty}

% for A4 size

\setlength\topmargin{-5mm} \setlength\oddsidemargin{-0cm}
\setlength\textheight{24.7cm} \setlength\textwidth{16cm}
\setlength\columnsep{0.6cm}  \newlength\titlebox \setlength\titlebox{2.00in}
\setlength\headheight{5pt}   \setlength\headsep{0pt}
%\setlength\footheight{0pt}
\setlength\footskip{1.0cm}
\setlength\leftmargin{0.0in}
%\thispagestyle{empty}
\pagestyle{empty}


\setlength{\parindent}{0in}
\setlength{\parskip}{2ex}

\begin{document}

\begin{center}
  {\Large \bf Introduction}
\end{center}

\vspace*{0.5cm}

%%%%%%%%%%%%%%%%%%%%%%%%%%%%%%%%%%%%%%%%%%%%%%%%%%%%%%%%%%%%%%%%%%%%%%%%

%%% INSERT YOUR INTRO HERE
Uralic is an interesting group of languages from the
computational-linguistic perspective.  The Uralic languages share large
parts of morphological and morphophonological complexity that is not present in
the Indo-European language family, which has traditionally dominated
computational-linguistic research.  This can be seen for example in number of
morphologically complex forms belonging to one word, which in Indo-European
languages is in range of ones or tens whereas for Uralic languages, it is in the
range of hundreds and thousands.  Furthermore, Uralic language situations share
a lot of geo-political aspects: the three national languages---Finnish, Estonian
and Hungarian---are comparably small languages and only moderately resourced in
terms of computational-linguistics while being stable and not in threat of
extinction. The recognised minority languages of western-European states, on the
other hand---such as North S\'{a}mi, Kven and V\~{o}ro---do clearly fall in the category
of lesser resourced and more threatened languages, whereas the majority of
Uralic languages in the east of Europe and Siberia are close to extinction.
Common to all rapid development of more advanced computational-linguistic
methods is required for continued vitality of the languages in everyday life, to
enable archiving and use of the languages with computers and other devices such
as mobile applications.

Computational linguistic Research inside Uralistics is being carried out only in
a handful of universities, research institutes and other sites and only by
relatively few researchers.  Our intention with organising this conference is to
gather these researchers from scattered institutions together in order to share
ideas and resources, and avoid duplicating efforts in gathering and enriching
these scarce resources. We want to initiate more concentrated effort in
collecting and improving language resources and technologies for the survival of
the Uralic languages and hope that our effort today will become an ongoing
tradition in the future.

For the current proceedings of The Third International Workshop on Computational
Linguistics for Uralic Languages, we accepted 10 high-quality submissions about
topics including computational lexicography, language documentation, optical
character recognition, dependency parsing, web-as-corpus as well as automatic
and rule-based morphological analysis methods. The covered languages are very
broad and reach from different S\'{a}mi languages, over Kven, Finnish, Komi, Udmurt,
Mari, Khanty, Mansi, and Tundra Nenets. Whereas some papers describe
language-specific research, others compare different languages or work on small
Uralic languages in general. These contributions are all very important for the
preservation and development of Uralic languages as well as for future
linguistic investigations on them.

The conference was organized in collaboration with The University of Oslo St.
Petersburg Representative Office and held in St. Petersburg, Russia, on January
23rd and 24th 2017.  The program consisted of an invited speech by Heiki-Jaan
Kaalep, a poster session, and four  talks during the first day and an open
discussion and individual project workshops during the second day.  The current
proceedings include the written versions all oral and poster presentations.\\

\noindent ---Tommi A Pirinen, Trond Trosterud, Francis M. Tyers, Michael Rie\ss ler\\
Conference organisers,\\
\today, St. Petersburg


\end{document}
