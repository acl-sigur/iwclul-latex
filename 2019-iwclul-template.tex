\documentclass[a4paper,notitlepage]{article}

% inlcude all these packages unless you have reason not to...
\usepackage{polyglossia}
\usepackage{fontspec}
\usepackage{xunicode}
\usepackage{xltxtra}
\usepackage{url}
\usepackage{hyperref}
\usepackage{expex}
\usepackage{natbib}

% Make licence footnote without marker
\makeatletter
\def\blfootnote{\gdef\@thefnmark{}\@footnotetext}
\makeatother

% Use a Free/Libre font with Finnish–Hungarian-Cyrillic-UPA coverage
\setmainfont[Mapping=tex-text]{Linux Libertine O}
% set languages to use
\setmainlanguage{english}
\setotherlanguages{finnish,russian}

% This makes it easier to create anonymised and final versions with one change:
\newif\ifcameraready
%%% *** Uncomment the following line for the for-review submission
%\camerareadyfalse
%%% *** Uncomment the following line for the for-publication submission
\camerareadytrue
%%% and comment the other line so as to only have one of the above statements
%%% uncommented

\begin{document}

\ifcameraready
\pagestyle{empty}
\fi

\title{Template for IWCLUL Workshop Proceedings\blfootnote{
    This work is licensed under a Creative Commons Attribution
    4.0 International Licence.  Licence details:
    \url{http://creativecommons.org/licenses/by/4.0/}
}
}

\ifcameraready
\author{First Author\\
University or Affiliation\\
Department or School\\
(Optionally Post Address)\\
\url{first.author@example.com} \and
Second Author\\
University or Affiliation\\
Department or School\\
(Optionally Post Address)\\
\url{second.author@example.com} \and
No Author Names\\
Should Appear on\\
The Review Submission\\
\url{it's\_anonymous!}
}
\fi

\date{}

\maketitle
\ifcameraready
\thispagestyle{empty}
\fi

\begin{abstract}
    Abstract can be between 150 and 300 words long. Separate keywords are not
    used. Reviewers use abstract to decide which articles to review and readers
    will use abstract to decide which articles to read, so make it informative
    and interesting. Remember to include an abstract in at least one uralic
    language.
\end{abstract}

{
\selectlanguage{finnish}
\begin{abstract}
    Abstrakti pitäisi kirjoittaa myös suomeksi tai jollain muulla uralilaisella
    kielellä.
\end{abstract}
}

{
\selectlanguage{russian}
\begin{abstract}
Вы также можете написать на русском языке
\end{abstract}
}


\section{Introduction}

This template is for authors submitting papers to the International Workshop on
Computational Linguistics for Uralic Languages. The papers to be submitted
should be in PDF format made using \XeLaTeX{} and use this tex template.  The
documentclass is \texttt{article} with \texttt{a4paper} option.  The standard
xelatex packages including polyglossia should be loaded for all papers. If
possible, use Linux Libertine (or Linux Libertine O) as mainfont.  This is a
freely available font with good range of unicode support and small-caps for
Leipzig glossing.  Authors who are unable to use freely available
\texttt{xelatex} should contact the conference organisers for other options.
Additional information and recent updates to this template are available from
\url{http://acl-sigur.github.io/}. The submissions are handled using easychair.
Please note that the published version will be processed using ACLPUB system
to make a publication in the ACL Anthology, the published version will
be scaled and resized.
Required tex packages should be in your texlive distribution, for ubuntu use:
\texttt{sudo apt-get install
texlive-\{xetex,latex-recommended,fonts-recommended\} fonts-linux-libertine}.


\section{Title page}

The title page should be as made with latex maketitle command. \textbf{Do not
include author names for the review version submission}, we use blind review
process. If you are using our latex template, just make sure to uncomment the
command \texttt{camerareadyfalse}.

To ease the work with licences and permissions, all submissions must retain
a footnote acknowledging that work is to be published under Creative
Commons Attribution 4.0 International Licence.

\section{Emphasis}

Follow common English typographical standards. The \emph{emphasis} is used for
sentence level highlights and \textbf{strong emphasis} for paragraph level
highlights.

\section{Glossing}

Linguistic glosses should follow Leipzig Glossing rules
\url{http://www.eva.mpg.de/lingua/resources/glossing-rules.php}, see (\nextx).

\ex
\begingl
\gla talo-i-ssa-ni //
\glb house-{\sc Pl}-{\sc Ine}-{\sc 1sg} //
\glft `in my houses' //
\endgl
\xe

For inline examples, use emphasis for original and parentheses for translation
\emph{talossasi} (in your house).


\section{Tables}

Tables like table~\ref{table:example}. should have caption underneath and
borders.

\begin{table}
    \center
    \begin{tabular}{|l|r|}
        \hline
        \bf Header & \bf Header \\
        \hline
        \bf Header & Data \\
        \hline
    \end{tabular}
    \caption{The rows are things and columns stuff and data is in percent units
    \label{table:example}}
\end{table}

\section{Citations}

Citations are made using bibtex: WALS~\citep{haspelmath2005world} is a good
resource.  It is possible to cite Russians using plain unicode, such as the
original version of \citet{levenshtein1965} algorithm.  Use bibliographystyle
\texttt{acl\_natbib}. If you get your bibtex snippets from google scholar, don't
forget to double check the data and add missing parts.  For software and
language technology resources, it is possible to use one of the modern
repository systems' citations, such as referring to omorfi\footnote{for freely
available and open source resources it is also recommended to give a link in
footnote: \url{https://github.com/flammie/omorfi}} using LINDAT repository like
this~\citep{omorfi}.

\section{Program code and command-lines}

Code examples, program listings and shell sessions can be shown as figures,
e.g. figure~\ref{code:analysis}.
Pseudo-code may be formatted using one of the relevant \LaTeX{} packages, e.g.,
\texttt{algorithm}.

\begin{figure}
    \center
    \begin{verbatim}
    $ hfst-lookup src/morphology.ftb3.hfst
    > talossani
    talossani talo+N+Sg+Ine+PxSg1
    \end{verbatim}
    \caption{Analysing Finnish from command-line with \texttt{hfst-lookup}
    \label{code:analysis}}
\end{figure}

\section{Mathematical formulas}

Inline formulas should use standard tex environment: $P(\mathrm{is}) >
P(\mathrm{unlikely})$. Pay attention to the use of roman font for text examples
and multicharacter variables and indexes. Long formulas should be numbered in
equation block:

\begin{equation}
    P(\hat x) = \frac{x+\alpha}{y+\alpha\times z}
\end{equation}

Remember to describe each function, variable and notation you use.

\section{Spelling}

Please try a splel-checker before submitting your texts.

\section{Camera-ready version}

Once you have been accepted for publication, please go through the reviewers
suggestions, and prepare a camera-ready version for publication. If you use our
latex template, simply uncomment the \texttt{camerareadytrue} command. The
camera ready version must not contain page numbering, must include the authors'
names and affiliations, and may contain an additional acknowledgments section.
Please upload the camera ready version as PDF to easychair by the given
deadline.

\ifcameraready
\section*{Acknowledgments}

Acknowledgments should be un-numbered last section. Do not include
acknowledgements in anonymised review version.
\fi

\bibliographystyle{acl_natbib}
\bibliography{iwclul}

\end{document}

